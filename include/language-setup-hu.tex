%--------------------------------------------------------------------------------------
% Elnevezések
%--------------------------------------------------------------------------------------
\newcommand{\bme}{Budapesti Műszaki és Gazdaságtudományi Egyetem}
\newcommand{\gpk}{Gépészmérnöki Kar}

\newcommand{\bmeatt}{Anyagtudomány és Technológia Tanszék}
\newcommand{\bmeara}{Áramlástan Tanszék}
\newcommand{\bmeenergia}{Energetikai Gépek és Rendszerek Tanszék}
\newcommand{\bmeepget}{Épületgépészeti és Gépészeti Eljárástechnika Tanszék}
\newcommand{\bmegtharom}{Gép- és Terméktervezés Tanszék}
\newcommand{\bmemanuf}{Gyártástudomány és -technológia Tanszék}
\newcommand{\bmehds}{Hidrodinamikai Rendszerek Tanszék}
\newcommand{\bmemogi}{Mechatronika, Optika és Gépészeti Informatika Tanszék}
\newcommand{\bmemm}{Műszaki Mechanika Tanszék}
\newcommand{\bmept}{Polimertechnika Tanszék}

\newcommand{\keszitette}{Készítette}
\newcommand{\konzulens}{Konzulens}
\newcommand{\temavezeto}{Témavezető}

\newcommand{\selectbsc}{
  \newcommand{\gpkmunkatipus}{Szakdolgozat}       % Dokumentum típusa [Document type]
  \newcommand{\gpkmunkatipusHU}{Szakdolgozat}     % Dokumentum típusa HU
  \newcommand{\gpkmunkatipusok}{Szakdolgozatok}   % többesszámban
  \newcommand{\gpkmunkatipustHU}{Szakdolgozatot}  % tárgyraggal
}
\newcommand{\selectmsc}{
  \newcommand{\gpkmunkatipus}{Diplomaterv}        % Dokumentum típusa [Document type]
  \newcommand{\gpkmunkatipusHU}{Diplomaterv}      % Dokumentum típusa HU
  \newcommand{\gpkmunkatipusok}{Diplomatervek}    % többesszámban
  \newcommand{\gpkmunkatipustHU}{Diplomatervet}   % tárgyraggal
}

\newcommand{\tdk}{TDK dolgozat}
\newcommand{\bsconlab}{BSc Önálló laboratórium}
\newcommand{\msconlabi}{MSc Önálló laboratórium 1.}
\newcommand{\msconlabii}{MSc Önálló laboratórium 2.}


\newcommand{\pelda}{Példa}
\newcommand{\definicio}{Definíció}
\newcommand{\tetel}{Tétel}

\newcommand{\jelolesek}{Jelölések jegyzéke}
\newcommand{\eloszo}{Előszó}
\newcommand{\bevezetes}{Bevezetés}
\newcommand{\koszonetnyilvanitas}{Köszönetnyilvánítás}
\newcommand{\osszefoglalas}{Összefoglalás}
\newcommand{\summary}{Summary}
\newcommand{\fuggelek}{Függelék}
\newcommand{\melleklet}{Mellékletek}

% Opcionálisan átnevezhető címek
\addto\captionsmagyar{%
\renewcommand{\listfigurename}{Illusztrációk jegyzéke}
%\renewcommand{\listtablename}{Saját táblázatjegyzék cím}
%\renewcommand{\bibname}{Saját irodalomjegyzék név}
}

\newcommand{\authorName}{\authorFamilyName{} \authorGivenName}
\newcommand{\consulentA}{\consulentATitle\consulentAFamilyName{} \consulentAGivenName}
\newcommand{\consulentB}{\consulentBTitle\consulentBFamilyName{} \consulentBGivenName}
\newcommand{\consulentC}{\consulentCTitle\consulentCFamilyName{} \consulentCGivenName}
\newcommand{\supervisor}{\supervisorTitle\supervisorFamilyName{}
\supervisorGivenName}

\newcommand{\selectthesislanguage}{\selecthungarian}
\newcommand{\selectforeignlanguage}{\selectenglish}

\bibliographystyle{huplain}

\def\lstlistingname{lista}

\newcommand{\appendixletter}{6} % a fofejezet-szamlalo az angol ABC 6. betuje (F) lesz
\newcommand{\annexletter}{13} % M betű
