%--------------------------------------------------------------------------------------
% Elnevezések
%--------------------------------------------------------------------------------------
\newcommand{\bme}{Budapesti Műszaki és Gazdaságtudományi Egyetem}
\newcommand{\gpk}{Gépészmérnöki Kar}

\newcommand{\bmeatt}{Anyagtudomány és Technológia Tanszék}
\newcommand{\bmeara}{Áramlástan Tanszék}
\newcommand{\bmeenergia}{Energetikai Gépek és Rendszerek Tanszék}
\newcommand{\bmeepget}{Épületgépészeti és Gépészeti Eljárástechnika Tanszék}
\newcommand{\bmegtharom}{Gép- és Terméktervezés Tanszék}
\newcommand{\bmemanuf}{Gyártástudomány és -technológia Tanszék}
\newcommand{\bmehds}{Hidrodinamikai Rendszerek Tanszék}
\newcommand{\bmemogi}{Mechatronika, Optika és Gépészeti Informatika Tanszék}
\newcommand{\bmemm}{Műszaki Mechanika Tanszék}
\newcommand{\bmept}{Polimertechnika Tanszék}

\newcommand{\keszitette}{Készítette}
\newcommand{\konzulens}{Konzulens}
\newcommand{\temavezeto}{Témavezető}

\newcommand{\bsc}{Szakdolgozat}
\newcommand{\bscrag}{o}                 % szakdolgozat[o]t, szakdolgozat[o]k 
\newcommand{\msc}{Diplomaterv}
\newcommand{\mscrag}{e}                 % diplomaterv[e]t, diplomaterv[e]k
\newcommand{\tdk}{TDK dolgozat}
\newcommand{\bsconlab}{BSc Önálló laboratórium}
\newcommand{\msconlabi}{MSc Önálló laboratórium 1.}
\newcommand{\msconlabii}{MSc Önálló laboratórium 2.}

\newcommand{\pelda}{Példa}
\newcommand{\definicio}{Definíció}
\newcommand{\tetel}{Tétel}

\newcommand{\bevezetes}{Bevezetés}
\newcommand{\koszonetnyilvanitas}{Köszönetnyilvánítás}
\newcommand{\fuggelek}{Függelék}
\newcommand{\melleklet}{Mellékletek}

% Opcionálisan átnevezhető címek
\addto\captionsmagyar{%
\renewcommand{\listfigurename}{Illusztrációk jegyzéke}
%\renewcommand{\listtablename}{Saját táblázatjegyzék cím}
%\renewcommand{\bibname}{Saját irodalomjegyzék név}
}

\newcommand{\szerzo}{\gpkszerzoVezeteknev{} \gpkszerzoKeresztnev}
\newcommand{\gpkkonzulensA}{\gpkkonzulensAMegszolitas\gpkkonzulensAVezeteknev{} \gpkkonzulensAKeresztnev}
\newcommand{\gpkkonzulensB}{\gpkkonzulensBMegszolitas\gpkkonzulensBVezeteknev{} \gpkkonzulensBKeresztnev}
\newcommand{\gpkkonzulensC}{\gpkkonzulensCMegszolitas\gpkkonzulensCVezeteknev{} \gpkkonzulensCKeresztnev}
\newcommand{\gpktemavezeto}{\gpktemavezetoMegszolitas\gpktemavezetoVezeteknev{}
\gpktemavezetoKeresztnev}

\newcommand{\selectthesislanguage}{\selecthungarian}

\bibliographystyle{huplain}

\def\lstlistingname{lista}

\newcommand{\appendixnumber}{6}  % a fofejezet-szamlalo az angol ABC 6. betuje (F) lesz
