%--------------------------------------------------------------------------------------
% Page layout setup
%--------------------------------------------------------------------------------------
% we need to redefine the pagestyle plain
% another possibility is to use the body of this command without \fancypagestyle
% and use \pagestyle{fancy} but in that case the special pages
% (like the ToC, the References, and the Chapter pages)remain in plane style

\pagestyle{plain}
\geometry{inner=30mm, outer=20mm, top=20mm, bottom=30mm}


%--------------------------------------------------------------------------------------
% Text and paragraph styling
%--------------------------------------------------------------------------------------

\sectionfont{\Large\upshape\bfseries}  % Section title font
\subsectionfont{\Large\itshape\mdseries}
\subsubsectionfont{\large\itshape\mdseries}
\setcounter{secnumdepth}{3}             % Section numbering depth

\sloppy                                 % Prevent text from spilling over the margin
\widowpenalty=10000 \clubpenalty=10000  % Prevent widow and oprhan rows
\def\hyph{-\penalty0\hskip0pt\relax}    % Enable hyphenation

\onehalfspacing                         % 1x Line spacing

% Text setup for Hungarian text
\newcommand{\selecthungarian}{
	\selectlanguage{magyar}
	\setlength{\parindent}{2em}			% Paragraph indentation
	\setlength{\parskip}{5pt}			% Paragraph spacing
	\frenchspacing
}

% Text setup for English text
\newcommand{\selectenglish}{
	\selectlanguage{english}
	\setlength{\parindent}{0em}
	\setlength{\parskip}{8pt}
	\nonfrenchspacing
}

%--------------------------------------------------------------------------------------
% Setup hyperref package
%--------------------------------------------------------------------------------------
\hypersetup{
    % bookmarks=true,            % show bookmarks bar?
    unicode=true,                % non-Latin characters in Acrobat's bookmarks
    pdfnewwindow=true,           % links in new window
    colorlinks=true,             % false: boxed links; true: colored links
    linkcolor=black,             % color of internal links
    citecolor=black,             % color of links to bibliography
    filecolor=black,             % color of file links
    urlcolor=black               % color of external links
}

%--------------------------------------------------------------------------------------
% Apply variables
%--------------------------------------------------------------------------------------
% This command is called in the main tex file and uses variables set there.
\newcommand{\applyvariables}{
	\author{\authorName}
	\title{\thesisTitle}

	\hypersetup{
		pdftitle={\thesisTitle},     % title
		pdfauthor={\authorName},     % author
		pdfsubject={\gpkmunkatipus}, % subject of the document
		pdfkeywords={\keywords},     % list of keywords (separate then by comma)
		pdfproducer={\authorName},   % producer of the document (organization)
		pdfcreator={LaTeX}           % creator of the document (application)
	}
}

%--------------------------------------------------------------------------------------
% Set up listings
%--------------------------------------------------------------------------------------
\definecolor{lightgray}{rgb}{0.95,0.95,0.95}
\lstset{
	basicstyle=\scriptsize\ttfamily, % print whole listing small
	keywordstyle=\color{black}\bfseries, % bold black keywords
	identifierstyle=, % nothing happens
	% default behavior: comments in italic, to change use
	% commentstyle=\color{green}, % for e.g. green comments
	stringstyle=\scriptsize,
	showstringspaces=false, % no special string spaces
	aboveskip=3pt,
	belowskip=3pt,
	backgroundcolor=\color{lightgray},
	columns=flexible,
	keepspaces=true,
	escapeinside={(*@}{@*)},
	captionpos=b,
	breaklines=true,
	frame=single,
	float=!ht,
	tabsize=2,
	literate=*
		{á}{{\'a}}1	{é}{{\'e}}1	{í}{{\'i}}1	{ó}{{\'o}}1	{ö}{{\"o}}1	{ő}{{\H{o}}}1	{ú}{{\'u}}1	{ü}{{\"u}}1	{ű}{{\H{u}}}1
		{Á}{{\'A}}1	{É}{{\'E}}1	{Í}{{\'I}}1	{Ó}{{\'O}}1	{Ö}{{\"O}}1	{Ő}{{\H{O}}}1	{Ú}{{\'U}}1	{Ü}{{\"U}}1	{Ű}{{\H{U}}}1
}


%--------------------------------------------------------------------------------------
% Set up theorem-like environments
%--------------------------------------------------------------------------------------
% Using ntheorem package -- see http://www.math.washington.edu/tex-archive/macros/latex/contrib/ntheorem/ntheorem.pdf

\theoremstyle{plain}
\theoremseparator{.}
\newtheorem{example}{\pelda}

\theoremseparator{.}
%\theoremprework{\bigskip\hrule\medskip}
%\theorempostwork{\hrule\bigskip}
\theorembodyfont{\upshape}
\theoremsymbol{{\large \ensuremath{\centerdot}}}
\newtheorem{definition}{\definicio}

\theoremseparator{.}
%\theoremprework{\bigskip\hrule\medskip}
%\theorempostwork{\hrule\bigskip}
\newtheorem{theorem}{\tetel}


%--------------------------------------------------------------------------------------
% Some new commands and declarations
%--------------------------------------------------------------------------------------
\newcommand{\code}[1]{{\upshape\ttfamily\scriptsize\indent #1}}
\newcommand{\doi}[1]{DOI: \href{http://dx.doi.org/\detokenize{#1}}{\raggedright{\texttt{\detokenize{#1}}}}} % A hivatkozások közt így könnyebb DOI-t megadni.

\DeclareMathOperator*{\argmax}{arg\,max}
%\DeclareMathOperator*[1]{\floor}{arg\,max}
\DeclareMathOperator{\sign}{sgn}
\DeclareMathOperator{\rot}{rot}


%--------------------------------------------------------------------------------------
% Setup captions
%--------------------------------------------------------------------------------------
\captionsetup[figure]{
	width=.75\textwidth,
	aboveskip=10pt}

\renewcommand{\captionlabelfont}{\it}
\renewcommand{\captionfont}{\footnotesize\it}

%--------------------------------------------------------------------------------------
% Hyphenation exceptions
%--------------------------------------------------------------------------------------
\hyphenation{Shakes-peare Mar-seilles ár-víz-tű-rő tü-kör-fú-ró-gép}


%--------------------------------------------------------------------------------------
% Command to exclude tables and images in the annex from the List of Figures/Tables
%--------------------------------------------------------------------------------------
\newcommand{\excludeFromLocAndLot}{
	% Redefine \addcontentsline to be silent when printing loc or toc entries
	\let\svaddcontentsline\addcontentsline
	\renewcommand\addcontentsline[3]{
		\ifthenelse{\equal{##1}{lof}}{}{
			\ifthenelse{\equal{##1}{lot}}{}{
				\svaddcontentsline{##1}{##2}{##3}
			}
		}
	}
}