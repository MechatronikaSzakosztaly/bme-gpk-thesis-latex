%--------------------------------------------------------------------------------------
% Dokumentum formátuma [Document format]
%--------------------------------------------------------------------------------------

%TODO Válaszd ki, hogy egyoldalas vagy kétoldalas legyen! [Choose format]
\documentclass[11pt,a4paper,oneside]{report}             % Egyoldalas [Single-side]
%\documentclass[11pt,a4paper,twoside,openright]{report}  % Kétoldalas [Duplex]


%--------------------------------------------------------------------------------------
% Csomagok inicializálása [Initializing packages]
%--------------------------------------------------------------------------------------
\usepackage{ifthen} % Used in macros

\usepackage[english,magyar]{babel} % Language support
\usepackage{geometry}
\usepackage{amsfonts,amsmath,amssymb} % Mathematical symbols
\usepackage{microtype} % Imrovements to typesetting
\usepackage{setspace} % For setting line spacing
\usepackage{cmap} % Enables more advenced text copying from the PDF document 
\usepackage{sectsty} % Section heading styling

\usepackage[unicode]{hyperref} % For hyperlinks in the generated document
\usepackage{booktabs} % For publication quality tables for LaTeX
\usepackage{graphicx} % For figure sizing
\usepackage[hang]{caption}
\usepackage{xcolor} % For code coloring in listings
\usepackage{listings} % For source code snippets
\usepackage[amsmath,thmmarks]{ntheorem} % Theorem-like environments

\usepackage[numbers]{natbib} % Bibliography

%\usepackage{fancyhdr} % For easy to use headers and footers

% thanks to http://tex.stackexchange.com/a/47579/71109
\usepackage{ifxetex}
\usepackage{ifluatex}
\newif\ifxetexorluatex % a new conditional starts as false
\ifnum 0\ifxetex 1\fi\ifluatex 1\fi>0
   \xetexorluatextrue
\fi

\ifxetexorluatex
  \usepackage{fontspec}
  % Palatino clone font (Tex Gyre Pagella) for text and math
  \usepackage{newpxmath}
  \setmainfont[Ligatures=TeX]{TeX Gyre Pagella}
\else
  \usepackage[T1]{fontenc}
  \usepackage[utf8]{inputenc}
  %\usepackage[lighttt]{lmodern} % Advanced version of the Computer Modern font
  % Palatino clone font (Tex Gyre Pagella) for text and math
  \usepackage{tgpagella, newpxmath}
\fi



%--------------------------------------------------------------------------------------
% Dokumentum nyelve [Language]
%--------------------------------------------------------------------------------------

%TODO Válaszd ki a nyelvet! [Select language]
%--------------------------------------------------------------------------------------
% Elnevezések
%--------------------------------------------------------------------------------------
\newcommand{\bme}{Budapesti Műszaki és Gazdaságtudományi Egyetem}
\newcommand{\gpk}{Gépészmérnöki Kar}

\newcommand{\bmeatt}{Anyagtudomány és Technológia Tanszék}
\newcommand{\bmeara}{Áramlástan Tanszék}
\newcommand{\bmeenergia}{Energetikai Gépek és Rendszerek Tanszék}
\newcommand{\bmeepget}{Épületgépészeti és Gépészeti Eljárástechnika Tanszék}
\newcommand{\bmegtharom}{Gép- és Terméktervezés Tanszék}
\newcommand{\bmemanuf}{Gyártástudomány és -technológia Tanszék}
\newcommand{\bmehds}{Hidrodinamikai Rendszerek Tanszék}
\newcommand{\bmemogi}{Mechatronika, Optika és Gépészeti Informatika Tanszék}
\newcommand{\bmemm}{Műszaki Mechanika Tanszék}
\newcommand{\bmept}{Polimertechnika Tanszék}

\newcommand{\keszitette}{Készítette}
\newcommand{\konzulens}{Konzulens}
\newcommand{\temavezeto}{Témavezető}

\newcommand{\selectbsc}{
  \newcommand{\gpkmunkatipus}{Szakdolgozat}       % Dokumentum típusa [Document type]
  \newcommand{\gpkmunkatipusHU}{Szakdolgozat}     % Dokumentum típusa HU
  \newcommand{\gpkmunkatipusok}{Szakdolgozatok}   % többesszámban
  \newcommand{\gpkmunkatipustHU}{Szakdolgozatot}  % tárgyraggal
}
\newcommand{\selectmsc}{
  \newcommand{\gpkmunkatipus}{Diplomaterv}        % Dokumentum típusa [Document type]
  \newcommand{\gpkmunkatipusHU}{Diplomaterv}      % Dokumentum típusa HU
  \newcommand{\gpkmunkatipusok}{Diplomatervek}    % többesszámban
  \newcommand{\gpkmunkatipustHU}{Diplomatervet}   % tárgyraggal
}

\newcommand{\tdk}{TDK dolgozat}
\newcommand{\bsconlab}{BSc Önálló laboratórium}
\newcommand{\msconlabi}{MSc Önálló laboratórium 1.}
\newcommand{\msconlabii}{MSc Önálló laboratórium 2.}

\newcommand{\szerzoijog}{Szerzői jog}

\newcommand{\pelda}{Példa}
\newcommand{\definicio}{Definíció}
\newcommand{\tetel}{Tétel}

\newcommand{\jelolesek}{Jelölések jegyzéke}
\newcommand{\eloszo}{Előszó}
\newcommand{\bevezetes}{Bevezetés}
\newcommand{\koszonetnyilvanitas}{Köszönetnyilvánítás}
\newcommand{\osszefoglalas}{Összefoglalás}
\newcommand{\summary}{Summary}
\newcommand{\fuggelek}{Függelék}
\newcommand{\melleklet}{Mellékletek}

% Opcionálisan átnevezhető címek
\addto\captionsmagyar{%
\renewcommand{\listfigurename}{Illusztrációk jegyzéke}
%\renewcommand{\listtablename}{Saját táblázatjegyzék cím}
%\renewcommand{\bibname}{Saját irodalomjegyzék név}
}

\newcommand{\authorName}{\authorFamilyName{} \authorGivenName}
\newcommand{\consulentA}{\consulentATitle\consulentAFamilyName{} \consulentAGivenName}
\newcommand{\consulentB}{\consulentBTitle\consulentBFamilyName{} \consulentBGivenName}
\newcommand{\consulentC}{\consulentCTitle\consulentCFamilyName{} \consulentCGivenName}
\newcommand{\supervisor}{\supervisorTitle\supervisorFamilyName{}
\supervisorGivenName}

\newcommand{\selectthesislanguage}{\selecthungarian}
\newcommand{\selectforeignlanguage}{\selectenglish}

\bibliographystyle{huplain}

\def\lstlistingname{lista}

\newcommand{\appendixletter}{6} % a fofejezet-szamlalo az angol ABC 6. betuje (F) lesz
\newcommand{\annexletter}{13} % M betű
 % Beállítások magyar nyelvű dolgozathoz
%%--------------------------------------------------------------------------------------
% Elnevezések
%--------------------------------------------------------------------------------------
\newcommand{\bme}{Budapest University of Technology and Economics}
\newcommand{\gpk}{Faculty of Mechanical Engineering}

\newcommand{\bmeatt}{Department of Materials Science and Engineering}
\newcommand{\bmeara}{Department of Fluid Mechanics}
\newcommand{\bmeenergia}{Department of Energy Engineering}
\newcommand{\bmeepget}{Department of Building Service Engineering and Process Engineering}
\newcommand{\bmegtharom}{Department of Machine and Product Design}
\newcommand{\bmemanuf}{Department of Manufacturing Science and Engineering}
\newcommand{\bmehds}{Department of Hydrodynamic Systems}
\newcommand{\bmemogi}{Department of Mechatornics, Optics and Mechanical Engineering Informatics}
\newcommand{\bmemm}{Department of Applied Mechanics}
\newcommand{\bmept}{Department of Polymer Engineering}

\newcommand{\keszitette}{Author}
\newcommand{\konzulens}{Advisor}
\newcommand{\temavezeto}{Supervisor}

\newcommand{\selectbsc}{
  \newcommand{\gpkmunkatipus}{Bachelor's Thesis}  % Dokumentum típusa [Document type]
  \newcommand{\gpkmunkatipusHU}{Szakdolgozat}     % Dokumentum típusa HU
  \newcommand{\gpkmunkatipusok}{Bachelor's Theses}% többesszámban
  \newcommand{\gpkmunkatipustHU}{Szakdolgozatot}  % tárgyraggal
}
\newcommand{\selectmsc}{
  \newcommand{\gpkmunkatipus}{Master's Thesis}    % Dokumentum típusa [Document type]
  \newcommand{\gpkmunkatipusHU}{Diplomaterv}      % Dokumentum típusa HU
  \newcommand{\gpkmunkatipusok}{Master's Theses}  % többesszámban
  \newcommand{\gpkmunkatipustHU}{Diplomatervet}   % tárgyraggal
}

\newcommand{\tdk}{Scientific Students' Association Report}
\newcommand{\bsconlab}{BSc Project Laboratory}
\newcommand{\msconlabi}{MSc Project Laboratory 1}
\newcommand{\msconlabii}{MSc Project Laboratory 2}

\newcommand{\szerzoijog}{Copyright}

\newcommand{\pelda}{Example}
\newcommand{\definicio}{Definition}
\newcommand{\tetel}{Theorem}

\newcommand{\jelolesek}{Symbols}
\newcommand{\eloszo}{Abstract}
\newcommand{\bevezetes}{Introduction}
\newcommand{\koszonetnyilvanitas}{Acknowledgements}
\newcommand{\osszefoglalas}{Summary}
\newcommand{\summary}{Összefoglalás}
\newcommand{\fuggelek}{Appendix}
\newcommand{\melleklet}{Annex}

\renewcommand{\figureautorefname}{Figure}
\renewcommand{\tableautorefname}{Table}
\renewcommand{\partautorefname}{Part}
\renewcommand{\chapterautorefname}{Chapter}
\renewcommand{\sectionautorefname}{Section}
\renewcommand{\subsectionautorefname}{Section}
\renewcommand{\subsubsectionautorefname}{Section}

% Optional custom titles
%\addto\captionsenglish{%
%\renewcommand*{\listfigurename}{Your list of figures title}
%\renewcommand*{\listtablename}{Your list of tables title}
%\renewcommand*{\bibname}{Your bibliography title}
%}

\newcommand{\authorName}{\authorGivenName{} \authorFamilyName}
\newcommand{\consulentA}{\consulentATitle\consulentAGivenName{} \consulentAFamilyName}
\newcommand{\consulentB}{\consulentBTitle\consulentBGivenName{} \consulentBFamilyName}
\newcommand{\consulentC}{\consulentCTitle\consulentCGivenName{} \consulentCFamilyName}
\newcommand{\supervisor}{\supervisorTitle\supervisorGivenName{} \supervisorFamilyName}

\newcommand{\selectthesislanguage}{\selectenglish}
\newcommand{\selectforeignlanguage}{\selecthungarian}

\bibliographystyle{plainnat}

\newcommand{\ie}{i.e.\@\xspace}
\newcommand{\Ie}{I.e.\@\xspace}
\newcommand{\eg}{e.g.\@\xspace}
\newcommand{\Eg}{E.g.\@\xspace}
\newcommand{\etal}{et al.\@\xspace}
\newcommand{\etc}{etc.\@\xspace}
\newcommand{\vs}{vs.\@\xspace}
\newcommand{\viz}{viz.\@\xspace} % videlicet
\newcommand{\cf}{cf.\@\xspace} % confer
\newcommand{\Cf}{Cf.\@\xspace}
\newcommand{\wrt}{w.r.t.\@\xspace} % with respect to
\newcommand{\approximately}{approx.\@\xspace}

\newcommand{\appendixletter}{1}  % a fofejezet-szamlalo az angol ABC 1. betuje (A) lesz
\newcommand{\annexletter}{2} % B betű
    % Settings for English document

% Megjegyzés: 
%         Ez a beállítás az automatikusan létrehozott címek, hivatkozások
%         nyelvét adja meg, valamint a nyelvre jellemző behúzási távolságot
%         használja a bekezdések elején.
%
% Note: 
%         This setting controls the language of generated titles and citations,
%         moreover the paragraph indentation.


%--------------------------------------------------------------------------------------
% Preambulum (LaTeX beállítások, makrók) [Preamble (LaTeX settings)]
%--------------------------------------------------------------------------------------
%--------------------------------------------------------------------------------------
% Page layout setup
%--------------------------------------------------------------------------------------
% we need to redefine the pagestyle plain
% another possibility is to use the body of this command without \fancypagestyle
% and use \pagestyle{fancy} but in that case the special pages
% (like the ToC, the References, and the Chapter pages)remain in plane style

\pagestyle{plain}
\geometry{inner=30mm, outer=20mm, top=20mm, bottom=30mm}


%--------------------------------------------------------------------------------------
% Text and paragraph styling
%--------------------------------------------------------------------------------------

\sectionfont{\Large\upshape\bfseries}  % Section title font
\subsectionfont{\Large\itshape\mdseries}
\subsubsectionfont{\large\itshape\mdseries}
\setcounter{secnumdepth}{3}             % Section numbering depth

\sloppy                                 % Prevent text from spilling over the margin
\widowpenalty=10000 \clubpenalty=10000  % Prevent widow and oprhan rows
\def\hyph{-\penalty0\hskip0pt\relax}    % Enable hyphenation

\onehalfspacing                         % 1.5x Line spacing

% Text setup for Hungarian text
\newcommand{\selecthungarian}{
	\selectlanguage{magyar}
	\setlength{\parindent}{2em}			% Paragraph indentation
	\setlength{\parskip}{5pt}			% Paragraph spacing
	\frenchspacing
}

% Text setup for English text
\newcommand{\selectenglish}{
	\selectlanguage{english}
	\setlength{\parindent}{0em}
	\setlength{\parskip}{8pt}
	\nonfrenchspacing
}

%--------------------------------------------------------------------------------------
% Setup hyperref package
%--------------------------------------------------------------------------------------
\hypersetup{
    % bookmarks=true,            % show bookmarks bar?
    unicode=true,                % non-Latin characters in Acrobat's bookmarks
    pdfnewwindow=true,           % links in new window
    colorlinks=true,             % false: boxed links; true: colored links
    linkcolor=black,             % color of internal links
    citecolor=black,             % color of links to bibliography
    filecolor=black,             % color of file links
    urlcolor=black               % color of external links
}

%--------------------------------------------------------------------------------------
% Apply variables
%--------------------------------------------------------------------------------------
% This command is called in the main tex file and uses variables set there.
\newcommand{\applyvariables}{
	\author{\authorName}
	\title{\thesisTitle}

	\hypersetup{
		pdftitle={\thesisTitle},     % title
		pdfauthor={\authorName},     % author
		pdfsubject={\gpkmunkatipus}, % subject of the document
		pdfkeywords={\keywords},     % list of keywords (separate then by comma)
		pdfproducer={\authorName},   % producer of the document (organization)
		pdfcreator={LaTeX}           % creator of the document (application)
	}
}

%--------------------------------------------------------------------------------------
% Set up listings
%--------------------------------------------------------------------------------------
\definecolor{lightgray}{rgb}{0.95,0.95,0.95}
\lstset{
	basicstyle=\scriptsize\ttfamily, % print whole listing small
	keywordstyle=\color{black}\bfseries, % bold black keywords
	identifierstyle=, % nothing happens
	% default behavior: comments in italic, to change use
	% commentstyle=\color{green}, % for e.g. green comments
	stringstyle=\scriptsize,
	showstringspaces=false, % no special string spaces
	aboveskip=3pt,
	belowskip=3pt,
	backgroundcolor=\color{lightgray},
	columns=flexible,
	keepspaces=true,
	escapeinside={(*@}{@*)},
	captionpos=b,
	breaklines=true,
	frame=single,
	float=!ht,
	tabsize=2,
	literate=*
		{á}{{\'a}}1	{é}{{\'e}}1	{í}{{\'i}}1	{ó}{{\'o}}1	{ö}{{\"o}}1	{ő}{{\H{o}}}1	{ú}{{\'u}}1	{ü}{{\"u}}1	{ű}{{\H{u}}}1
		{Á}{{\'A}}1	{É}{{\'E}}1	{Í}{{\'I}}1	{Ó}{{\'O}}1	{Ö}{{\"O}}1	{Ő}{{\H{O}}}1	{Ú}{{\'U}}1	{Ü}{{\"U}}1	{Ű}{{\H{U}}}1
}


%--------------------------------------------------------------------------------------
% Set up theorem-like environments
%--------------------------------------------------------------------------------------
% Using ntheorem package -- see http://www.math.washington.edu/tex-archive/macros/latex/contrib/ntheorem/ntheorem.pdf

\theoremstyle{plain}
\theoremseparator{.}
\newtheorem{example}{\pelda}

\theoremseparator{.}
%\theoremprework{\bigskip\hrule\medskip}
%\theorempostwork{\hrule\bigskip}
\theorembodyfont{\upshape}
\theoremsymbol{{\large \ensuremath{\centerdot}}}
\newtheorem{definition}{\definicio}

\theoremseparator{.}
%\theoremprework{\bigskip\hrule\medskip}
%\theorempostwork{\hrule\bigskip}
\newtheorem{theorem}{\tetel}


%--------------------------------------------------------------------------------------
% Some new commands and declarations
%--------------------------------------------------------------------------------------
\newcommand{\code}[1]{{\upshape\ttfamily\scriptsize\indent #1}}
\newcommand{\doi}[1]{DOI: \href{http://dx.doi.org/\detokenize{#1}}{\raggedright{\texttt{\detokenize{#1}}}}} % A hivatkozások közt így könnyebb DOI-t megadni.

\DeclareMathOperator*{\argmax}{arg\,max}
%\DeclareMathOperator*[1]{\floor}{arg\,max}
\DeclareMathOperator{\sign}{sgn}
\DeclareMathOperator{\rot}{rot}


%--------------------------------------------------------------------------------------
% Setup captions
%--------------------------------------------------------------------------------------
\captionsetup[figure]{
	width=.75\textwidth,
	aboveskip=10pt}

\renewcommand{\captionlabelfont}{\it}
\renewcommand{\captionfont}{\footnotesize\it}

%--------------------------------------------------------------------------------------
% Hyphenation exceptions
%--------------------------------------------------------------------------------------
\hyphenation{Shakes-peare Mar-seilles ár-víz-tű-rő tü-kör-fú-ró-gép}


%--------------------------------------------------------------------------------------
% Command to exclude tables and images in the annex from the List of Figures/Tables
%--------------------------------------------------------------------------------------
\newcommand{\excludeFromLocAndLot}{
	% Redefine \addcontentsline to be silent when printing loc or toc entries
	\let\svaddcontentsline\addcontentsline
	\renewcommand\addcontentsline[3]{
		\ifthenelse{\equal{##1}{lof}}{}{
			\ifthenelse{\equal{##1}{lot}}{}{
				\svaddcontentsline{##1}{##2}{##3}
			}
		}
	}
}


%--------------------------------------------------------------------------------------
% Munkatípus [Thesis type]
%--------------------------------------------------------------------------------------

%TODO Válaszd ki a munkatípust [Select thesis type]
\selectbsc    % Szakdolgozat [Bachelor's]
%\selectmsc   % Diplomaterv [Master's]


%--------------------------------------------------------------------------------------
% Változók beállítása [Setting variables]
%--------------------------------------------------------------------------------------

%TODO Állítsd be az alábbi változókat [Set these variables]

% Szerző [Author]
\def\authorFamilyName{Mogi}
\def\authorGivenName{Róbert}
\def\neptun{ABC123}

% Konzulens 1 [Consulent 1]
\def\consulentATitle{dr.~}
\def\consulentAFamilyName{Konzulens}
\def\consulentAGivenName{Egy}
\def\consulentARank{mestertanár}

% Konzulens 2 [Consulent 2], ha nincs hagyd üresen
\def\consulentBTitle{}
\def\consulentBFamilyName{Konzulens}
\def\consulentBGivenName{Kettő}
\def\consulentBRank{okl. gépészmérnök}

% Konzulens 3 [Consulent 3], ha nincs hagyd üresen 
\def\consulentCTitle{}
\def\consulentCFamilyName{}
\def\consulentCGivenName{}
\def\consulentCRank{}

% Témavezető
\def\supervisorTitle{dr.~}
\def\supervisorFamilyName{Témavezető}
\def\supervisorGivenName{Egy}
\def\supervisorRank{adjunktus}

% Dolgozat címe [Thesis title]
\def\thesisTitle{Gazdájához kötődő PID-szabályozó tervezése}

% Kulcsszavak (a PDF-be) [Keywords (to PDF)]
\def\keywords{mechatronika, szabályozástechnika, etorobotika, ipar 4.0}

% Tanszék [Department]
%TODO Válassz az alábbiak közül
% \bmeatt    \bmeara  \bmeenergia  \bmeepget  \bmegtharom 
% \bmemanuf  \bmehds  \bmemogi     \bmemm     \bmept
\def\department{\bmemogi}

%TODO Cseréld le a figures/tanszek_logo.pdf képet a tanszéked logójára!
%     [Replace figures/tanszek_logo.pdf with the logo of your department]

% Elzártan kezelendő dolgozat [Restricted access]
%TODO Töltsd ki a korlátozás lejártának dátumát! 
%     [Fill in the end date of the restriction]
\def\endOfRestrictedAccess{... év ... hónap ... nap}


% Változók beállítása a PDF fájlhoz [Apply variables for the PDF file]
\applyvariables


%--------------------------------------------------------------------------------------
% Dokumentum törzse [Document body]
%--------------------------------------------------------------------------------------

\begin{document}

\pagenumbering{gobble}
\selectthesislanguage

% Címoldal [Titlepage]
\hypersetup{pageanchor=false}

%--------------------------------------------------------------------------------------
% Szennycímoldal [Cover title page]
%--------------------------------------------------------------------------------------

\clearpage
\begin{center}
\MakeUppercase{\authorName}\\[0.1cm]
\MakeUppercase{\gpkmunkatipus}\\[0.1cm]
\end{center}
\thispagestyle{empty}

%--------------------------------------------------------------------------------------
% Sorozatcímoldal [Series title page]
%--------------------------------------------------------------------------------------
\clearpage
\begin{center}

\includegraphics[width=60mm,keepaspectratio]{figures/bme_logo.pdf}\\
\vspace{0.3cm}
\MakeUppercase{\textbf{\bme}}\\[0.1cm]
\MakeUppercase{\textmd{\gpk}}\\[0.1cm]
\MakeUppercase{\textmd{\department}}\\[0.8cm]

\includegraphics[height=40mm,keepaspectratio]{figures/tanszek_logo}\\[0.5cm]
\MakeUppercase{\gpkmunkatipusok}

\end{center}
\thispagestyle{empty}

%--------------------------------------------------------------------------------------
% Címoldal [Title page]
%--------------------------------------------------------------------------------------
\begin{titlepage}
\begin{center}
\includegraphics[width=60mm,keepaspectratio]{figures/bme_logo.pdf}\\
\vspace{0.3cm}
\MakeUppercase{\textbf{\bme}}\\[0.1cm]
\MakeUppercase{\textmd{\gpk}}\\[0.1cm]
\MakeUppercase{\textmd{\department}}

\vspace{4.0cm}
{\huge \textsc{\authorName}}\\[0.8cm]
{\huge \MakeUppercase{\gpkmunkatipus}}\\[0.8cm]
{\Large \thesisTitle}

\vspace{3.0cm}

{
	\renewcommand{\arraystretch}{0.85}
	\begin{tabular}{ll}
	 \makebox[7cm][l]{\konzulens:} & \makebox[7cm][l]{\temavezeto:} \\
	 \noalign{\smallskip}
	 \makebox[7cm][l]{\hspace{1cm}\emph{\consulentA}} & \makebox[7cm][l]{\hspace{1cm}\emph{\supervisor}} \\
	 \makebox[7cm][l]{\hspace{1cm}\consulentARank} & \makebox[7cm][l]{\hspace{1cm}\supervisorRank} \\
	 \\
	 \makebox[7cm][l]{\hspace{1cm}\emph{\consulentB}} & \\
	 \makebox[7cm][l]{\hspace{1cm}\consulentBRank} & \\
	 \\
	 \makebox[7cm][l]{\hspace{1cm}\emph{\consulentC}} & \\
	 \makebox[7cm][l]{\hspace{1cm}\consulentCRank} & \\
	 
	\end{tabular}
}

\vfill
{\Large Budapest, \the\year.}
\end{center}
\end{titlepage}
\hypersetup{pageanchor=false}
\thispagestyle{empty}
  % Szakdolgozat/Diplomaterv címlap [Thesis]


% Copytightoldal [Copyright page]
%TODO Válaszd ki a megfelelőt! [Choose one]
\selectlanguage{magyar}
\pagenumbering{gobble}
\selecthungarian
%--------------------------------------------------------------------------------------
% Copyrightoldal
%--------------------------------------------------------------------------------------
\begin{flushleft}
Szerzői jog {\textcopyright} \szerzo, \the\year.
\end{flushleft}

\vfill
\clearpage
\thispagestyle{empty} % an empty page

\selectthesislanguage
               % Nyílt kezelésű [Open access]
%\selectlanguage{magyar}
\pagenumbering{gobble}
\selecthungarian
%--------------------------------------------------------------------------------------
% Copyrightoldal
%--------------------------------------------------------------------------------------
\begin{flushleft}
Szerzői jog {\textcopyright} \authorName, \the\year.
\end{flushleft}

\vspace{0.5cm}

\begin{center}
\textbf{ZÁRADÉK}\\
\end{center}

\vspace{0.5cm}
\noindent
Ez a \MakeLowercase{\gpkmunkatipusHU} elzártan kezelendő és őrzendő, a hozzáférése a vonatkozó szabályok szerint korlátozott, a dolgozat tartalmát csak az arra feljogosított személyek ismerhetik.

A korlátozott hozzáférés időtartamának lejártáig az arra feljogosítottakon kívül csak a korlátozást kérelmező személy vagy gazdálkodó szervezet írásos engedélyéjével rendelkező személy nyerhet betekintést a dolgozat tartalmába.

\vspace{0.3cm}

%TODO: fill out the date
A hozzáférés korlátozása és a zárt kezelés \endOfRestrictedAccess ján ér véget.

\vfill
\clearpage
\thispagestyle{empty} % an empty page

\selectthesislanguage
   % Elzárt kezelésű [Restricted access]


% Feladatkiírás [Project page]
%TODO A nyomtatott verzóban ne szerepeljen! [Remove before printing]
%--------------------------------------------------------------------------------------
% Feladatkiiras (a tanszeken atveheto, kinyomtatott valtozat)
%--------------------------------------------------------------------------------------
\begin{center}
\large
\textbf{FELADATKIÍRÁS}\\
\end{center}
\thispagestyle{empty}

A feladatkiírást a tanszéki adminisztrációban lehet átvenni, és a leadott munkába eredeti, tanszéki pecséttel ellátott és a tanszékvezető által aláírt lapot kell belefűzni (ezen oldal \emph{helyett}, ez az oldal csak útmutatás). Az elektronikusan feltöltött dolgozatban már nem kell beleszerkeszteni ezt a feladatkiírást.



% Nyilatkozatok [Declarations]
\selectlanguage{magyar}
\selecthungarian
\pagenumbering{roman}
\setcounter{page}{6}
\cleardoublepage % duplexnél páratlan oldalon legyen
%--------------------------------------------------------------------------------------
% Nyilatkozatok
%--------------------------------------------------------------------------------------
\begin{center}
\section*{NYILATKOZATOK}
\end{center}

\vspace{0.5cm}

\begin{center}
\emph{Beadhatósági nyilatkozat}
\end{center}
A jelen \MakeLowercase{\gpkmunkatipusHU} az üzem/intézmény által elvárt szakmai színvonalnak mind tartalmilag, mind formailag megfelel, beadható.

\begin{flushleft}
Kelt,
\end{flushleft}

\begin{flushright}
 \makebox[7cm][l]{Az üzem részéről:}\\
 \vspace{0.5cm}
 \makebox[7cm]{\rule{6cm}{.4pt}}\\
 \makebox[7cm]{\emph{üzemi konzulens}}
\end{flushright}
\vspace{0.6cm}

%--------------------------------------------------------------------------------------

\begin{center}
\emph{Elfogadási nyilatkozat}
\end{center}
Ezen \MakeLowercase{\gpkmunkatipusHU} a Budapesti Műszaki és Gazdaságtudományi Egyetem Gépészmérnöki Kara által a Diplomatervezési és Szakdolgozat feladatokra előírt valamennyi tartalmi és formai követelménynek, továbbá a feladatkiírásban előírtaknak maradéktalanul eleget tesz. E \MakeLowercase{\gpkmunkatipustHU} a nyilvános bírálatra és nyilvános előadásra alkalmasnak tartom.

\begin{flushleft}
A beadás időpontja:
\end{flushleft}

\begin{flushright}
 \makebox[7cm]{\rule{6cm}{.4pt}}\\
 \makebox[7cm]{\emph{témavezető}}
\end{flushright}
\vspace{0.6cm}

%--------------------------------------------------------------------------------------

\begin{center}
\emph{Nyilatkozat az önálló munkáról}
\end{center}
Alulírott,  \emph{\authorFamilyName{} \authorGivenName} (\neptun), a Budapesti Műszaki és Gazdaságtudományi Egyetem hallgatója, büntetőjogi és fegyelmi felelősségem tudatában kijelentem és sajátkezű aláírásommal igazolom, hogy ezt a \MakeLowercase{\gpkmunkatipustHU} meg nem engedett segítség nélkül, saját magam készítettem, és dolgozatomban csak a megadott forrásokat használtam fel. Minden olyan részt, melyet szó szerint vagy azonos értelemben, de átfogalmazva más forrásból átvettem, egyértelműen, a hatályos előírásoknak megfelelően, a forrás megadásával megjelöltem.

\begin{flushleft}
Budapest, \today
\end{flushleft}

\begin{flushright}
 \makebox[7cm]{\rule{6cm}{.4pt}}\\
 \makebox[7cm]{\emph{hallgató}}
\end{flushright}


\vfill
\clearpage

\selectthesislanguage

\newcounter{romanPage}
\setcounter{romanPage}{\value{page}}
\stepcounter{romanPage}


\selectthesislanguage
% Tartalomjegyzék [Table of Contents]
\setcounter{tocdepth}{3}  % Tartalomjegyzék mélysége [ToC depth]
\tableofcontents\vfill


% Ábrák és táblázatok jegyzéke [List of Figures, Tables]
%TODO Kommenteld ki, ha használni szeretnéd. [Uncomment to use]
%\listoffigures\addcontentsline{toc}{chapter}{\listfigurename}   % Ábrák jegyzéke - opcionális
%\listoftables\addcontentsline{toc}{chapter}{\listtablename}     % Táblázatok jegyzéke - opcionális


% Előszó [Preface]
%----------------------------------------------------------------------------
\chapter*{\eloszo}\addcontentsline{toc}{chapter}{\eloszo}
%----------------------------------------------------------------------------

Az előszó legtöbbször személyes hangú, eligazító jellegű írás, amely a mű megírásának okairól, születésének körülményeiről szól. Az előszó nem szerves része a főszövegnek, hanem annak kiegészítése.
Ugyancsak az előszóban fejtheti ki a szerző a mű megértéséhez szükséges szempontokat, a követett módszereket, utalhat a fontosabb előzményekre és szakirodalomra.
Az előszó ne legyen terjedelmes.


Jelen dokumentum egy diplomaterv-sablon, amely formai keretet ad a BME Gépészmérnöki Karán végző hallgatók által elkészítendő szakdolgozatnak és diplomatervnek. A sablon használata opcionális. Ez a sablon \LaTeX~alapú, a \emph{TeXLive} \TeX-implementációval és a PDF-\LaTeX~fordítóval működőképes.
A sablon forrása a Mechatronika Szakosztály GitHub tárhelyén\footnotemark{} elérhető. Amennyiben hibát találtál, vagy észrevételed, javaslatod lenne, kérlek ott jelezd.

\footnotetext{\url{https://github.com/MechatronikaSzakosztaly/bme-gpk-thesis-latex}}

\begin{center}
    $\thicksim \; \thicksim \; \thicksim$
\end{center}


\subsubsection*{Köszönetnyilvánítás}
\emph{A köszönetnyilvánítás ide írható.} Ez a sablon a Villamosmérnöki és Informatikai Kar Méréstechnika és Információs Rendszerek Tanszék szakdolgozat és diplomaterv sablonja alapján készült. Köszönöm készítőinek és karbantartóinak a munkájukat.


\vspace{0.5cm}

\begin{flushleft}
{Budapest, \today}
\end{flushleft}

\begin{flushright}
\emph{\authorName}
\end{flushright}

\vfill



% Jelölések jegyzéke [Table of Symbols]
\newcommand{\tss}{\textsuperscript}     % tss = felső index
%----------------------------------------------------------------------------
\chapter*{\jelolesek}\addcontentsline{toc}{chapter}{\jelolesek}
%----------------------------------------------------------------------------

A táblázatban a többször előforduló jelölések magyar és angol nyelvű elnevezése, 
valamint a fizikai mennyiségek esetén annak mértékegysége található. Az egyes 
mennyiségek jelölése – ahol lehetséges – megegyezik hazai és a nemzetközi 
szakirodalomban elfogadott jelölésekkel. A ritkán alkalmazott jelölések 
magyarázata első előfordulási helyüknél található.

%~~~~~~~~~~~~~~~~~~~~~~~~~~~~~~~~~~~~~~~~~~~~~~~~~~~~~~~~~~~~~~~~~~~~~~~~~~~~~~~~~~~~~
% A táblázatokat ABC rendben kell feltölteni, először mindig a kisbetűvel
% kezdve. Ha egyazon betűjelnek több értelmezése is van, akkor mindegyiket kü-
% lön sorban kell feltüntetni. Konstansok esetén az értéket is a táblázatba
% kell írni.
% Dimenzió nélküli mennyiségek mértékegysége 1 és nem: – !
% A jelölésjegyzékben csak SI vagy SI-n kívüli engedélyezett mértékegységeket
% szabad feltüntetni. Egy dokumentumon belül az SI és pl. az angolszász
% mértékrendszer nem keverhető!
%~~~~~~~~~~~~~~~~~~~~~~~~~~~~~~~~~~~~~~~~~~~~~~~~~~~~~~~~~~~~~~~~~~~~~~~~~~~~~~~~~~~~~

%~~~~~~~~~~~~~~~~~~~~~~~~~~~~~~~~~~~~~~~~~~~~~~~~~~~~~~~~~~~~~~~~~~~~~~~~~~~~~~~~~~~~~
% A Jelölés oszlop alapvetően kurzív betűváltozattal szedendő, a Mértékegység
% oszlopot álló betűkkel kell szedni. Felső indexhez használható a \tss{}
% parancs.
%~~~~~~~~~~~~~~~~~~~~~~~~~~~~~~~~~~~~~~~~~~~~~~~~~~~~~~~~~~~~~~~~~~~~~~~~~~~~~~~~~~~~~

\def\arraystretch{1.5}%  vertical cell padding

\subsubsection*{Latin betűk}
\begin{center}
    \begin{tabular}{lp{10cm}l}
        \hline
        Jelölés & Megnevezés, megjegyzés, érték & Mértékegység \\ 
        \hline
        $g$     & gravitációs gyorsulás (9.81)  & m/s\tss{2}     \\
        $p$     & nyomás                        & bar           \\
        $s$     & fajlagos entrópia             & J/(kg$\cdot$K)\\
        \hline
    \end{tabular}    
\end{center}



\subsubsection*{Görög betűk}
\begin{center}
    \begin{tabular}{lp{10cm}l}
        \hline
        Jelölés & Megnevezés, megjegyzés, érték & Mértékegység \\ 
        \hline
        $\eta$  & hatásfok                      & 1             \\      
        $\rho$  & sűrűség                       & kg/m\tss{3}    \\
        \hline
    \end{tabular}
\end{center}



\subsubsection*{Indexek, kitevők}
\begin{center}
    \begin{tabular}{lp{12.8cm}}
        \hline
        Jelölés & Megnevezés, értelmezés\\ 
        \hline
        $i$     & általános futóindex (egész szám)  \\
        nom     & névleges (nominális) érték        \\
        opt     & legkedvezőbb (optimális) érték    \\
        \hline
    \end{tabular}    
\end{center}


\def\arraystretch{1}%  vertical cell padding



% Főszöveg [The main part of the thesis]
\cleardoublepage
\pagenumbering{arabic}
%TODO Importáld a saját fejezeteidet [Import your own content]
\include{chapters/introduction}      % Bevezetés
\include{chapters/latex-tools}       % 2. fejezet
\include{chapters/thesis-format}     % 3. fejezet
\include{chapters/template-usage}    % 4. fejezet
%----------------------------------------------------------------------------
\chapter{\osszefoglalas} % (Eredmények értékelése)
%----------------------------------------------------------------------------


\section{Eredmények}
Az összefoglaló értékelés a három oldalt lehetőleg ne haladja meg! 
Az elvégzett munka és eredményeinek bemutatása egyes szám első személyben fogalmazva.


\section{Javaslatok/Következtetések/Tanulságok} % Válassz egyet
A feladat elkészítése során levont tanulságok összefoglalása. Javaslattétel, 
továbbfejlesztési lehetősége bemutatása, előretekintés a jövőbe stb.

% Keltezés, aláírás
\vspace{0.5cm}

\begin{flushleft}
{Budapest, \today}
\end{flushleft}

\begin{flushright}
\emph{\authorName}
\end{flushright}

\vfill
           % Összefoglalás


% Bibliográfia [Bibliography]
\addcontentsline{toc}{chapter}{\bibname}
\bibliography{bib/mybib}


% Idegen nyelvű (angol) összefoglaló [Foreign language summary]
%----------------------------------------------------------------------------
\chapter*{\summary}\addcontentsline{toc}{chapter}{\summary}
%----------------------------------------------------------------------------

\selectforeignlanguage % angol (magyar) nyelvi beállítások

Az elvégzett munka rövid, másfél oldalt meg nem haladó, de legalább 2/3 
oldalnyi terjedelmű angol nyelvű összefoglalása.

Angol nyelven készített dolgozat esetén magyar nyelvű összefoglaló kell, ha a
készítő magyar anyanyelvű. Nem angol vagy nem magyar nyelven készített
dolgozat esetén kötelező az angol nyelvű összefoglaló, és ha a készítő magyar
anyanyelvű, akkor a magyar nyelvű is.

\vspace{0.5cm}
\paragraph{Keywords} \emph{\keywords}  % A kulcsszavak a fő tex fájlban vannak definiálva


\selectthesislanguage % térjünk vissza magyar (angol) nyelvre



% Függelék és mellékletek [Appendices]
\excludefromlocandtoc % Az ábrákat és a táblázatokat hagyja ki a jegyzékből

%----------------------------------------------------------------------------
\appendix
%----------------------------------------------------------------------------
\chapter*{\fuggelek}\addcontentsline{toc}{chapter}{\fuggelek}
%----------------------------------------------------------------------------
\setcounter{chapter}{\appendixletter} % F betű
%\setcounter{equation}{0}
\numberwithin{equation}{section}
\numberwithin{figure}{section}
\numberwithin{lstlisting}{section}
%\numberwithin{tabular}{section}


% Ismertető - töröld ki
%----------------------------------------------------------------------------
A függelék a főszöveget kiegészíti további részletezéssel. Ide kerül minden kiegészítő információ, ami nem tartozik szorosan a feladat témájához. A függelék rendszerint nem önálló dokumentum. A főszöveg általában nem hivatkozik rá. Általában saját munka.

A dolgozat opcionális eleme, csak igény esetén kell használni.
%----------------------------------------------------------------------------

%----------------------------------------------------------------------------
\section{A TeXstudio felülete}
%----------------------------------------------------------------------------
\begin{figure}[!ht]
\centering
\includegraphics[width=150mm, keepaspectratio]{figures/TeXstudio.png}
\caption{A TeXstudio \LaTeX-szerkesztő.} 
\end{figure}

%----------------------------------------------------------------------------
\clearpage\section{Válasz az ,,Élet, a világmindenség, meg minden'' kérdésére}
%----------------------------------------------------------------------------
A Pitagorasz-tételből levezetve
\begin{align}
c^2=a^2+b^2=42.
\end{align}
A Faraday-indukciós törvényből levezetve
\begin{align}
\rot E=-\frac{dB}{dt}\hspace{1cm}\longrightarrow \hspace{1cm}
U_i=\oint\limits_\mathbf{L}{\mathbf{E}\mathbf{dl}}=-\frac{d}{dt}\int\limits_A{\mathbf{B}\mathbf{da}}=42.
\end{align}
        % Függelék - opcionális
%----------------------------------------------------------------------------
\appendix
%----------------------------------------------------------------------------
\chapter*{\melleklet}\addcontentsline{toc}{chapter}{\melleklet}
%----------------------------------------------------------------------------
\setcounter{chapter}{\annexletter} % M betű
\setcounter{section}{0}
%\setcounter{equation}{0}
\numberwithin{equation}{section}
\numberwithin{figure}{section}
\numberwithin{lstlisting}{section}
%\numberwithin{tabular}{section}

%----------------------------------------------------------------------------
\section{Első melléklet}
%----------------------------------------------------------------------------

A melléklet(ek)ben olyan információkat célszerű elhelyezni, melyek nélkül a főszövegben közöltek nem értelmezhetők, de az ott történő elhelyezésük jelentősen megnövelné a főszöveg terjedelmét. A melléklet rendszerint önállóan is értelmezhető dokumentum. Gyakran nem az író saját munkája. 

A mellékletbe kerülnek például a
\begin{itemize}
  \item terjedelmes, sok adatot tartalmazó táblázatok,
  \item mérési és egyéb jegyzőkönyvek,
  \item levél és faxmásolatok,
  \item szerződésmásolatok,
  \item műszaki rajzok és műszaki leírások,
  \item nagyméretű, ill. összetett kapcsolási vázlatok,
  \item térképek,
  \item fényképek. 
\end{itemize}

A mellékletbe kerülő információ mennyisége esetenként szükségessé teheti több 
melléklet kialakítását. Ebben az esetben célszerű lehet azt például a fenti 
csoportosításban kialakítani, és az egyes mellékleteket folyószámmal ellátni 
(1. Melléklet, 2. Melléklet stb.).
A melléklet tipográfiai kialakítására ugyanazok a szabályok vonatkoznak mint a 
függelékére. 
Minden, a mellékletben található információhordozót (táblázat, ábra, 
fénykép stb.) egyedi számmal és címmel kell ellátni. Ezt az azonosítót kell 
használni a főszövegben az ezekre való hivatkozások során.


%----------------------------------------------------------------------------
\clearpage
\section{Kapcsolási rajzok}
%----------------------------------------------------------------------------

% Képfelirattal ellátott kapcsolási rajzok, táblázatok, fényképek...

\begin{figure}[!ht]
\centering
\includegraphics[width=0.9\textwidth, keepaspectratio]{figures/uMogi.pdf}
\caption{A uMogi kapcsolási rajza.} 
\end{figure}
           % Mellékletek

\end{document}
