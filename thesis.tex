% !TeX spellcheck = hu_HU
% !TeX encoding = UTF-8
% !TeX program = pdflatex
% TODO [Change language to en_GB (recommended) or en_US for English documents]
\documentclass[11pt,a4paper,oneside]{report}             % Egyoldalas [Single-side]
%\documentclass[11pt,a4paper,twoside,openright]{report}  % Kétoldalas [Duplex]


%--------------------------------------------------------------------------------------
% Változók beállítása [Setting variables]
%--------------------------------------------------------------------------------------

\usepackage{ifthen} % Used in macros

\usepackage[english,magyar]{babel} % Language support
\usepackage{geometry}
\usepackage{amsfonts,amsmath,amssymb} % Mathematical symbols
\usepackage{microtype} % Imrovements to typesetting
\usepackage{setspace} % For setting line spacing
\usepackage{cmap} % Enables more advenced text copying from the PDF document 
\usepackage{sectsty} % Section heading styling

\usepackage[unicode]{hyperref} % For hyperlinks in the generated document
\usepackage{booktabs} % For publication quality tables for LaTeX
\usepackage{graphicx} % For figure sizing
\usepackage[hang]{caption}
\usepackage{xcolor} % For code coloring in listings
\usepackage{listings} % For source code snippets
\usepackage[amsmath,thmmarks]{ntheorem} % Theorem-like environments

\usepackage[numbers]{natbib} % Bibliography

%\usepackage{fancyhdr} % For easy to use headers and footers

% thanks to http://tex.stackexchange.com/a/47579/71109
\usepackage{ifxetex}
\usepackage{ifluatex}
\newif\ifxetexorluatex % a new conditional starts as false
\ifnum 0\ifxetex 1\fi\ifluatex 1\fi>0
   \xetexorluatextrue
\fi

\ifxetexorluatex
  \usepackage{fontspec}
  % Palatino clone font (Tex Gyre Pagella) for text and math
  \usepackage{newpxmath}
  \setmainfont[Ligatures=TeX]{TeX Gyre Pagella}
\else
  \usepackage[T1]{fontenc}
  \usepackage[utf8]{inputenc}
  %\usepackage[lighttt]{lmodern} % Advanced version of the Computer Modern font
  % Palatino clone font (Tex Gyre Pagella) for text and math
  \usepackage{tgpagella, newpxmath}
\fi


%TODO Állítsd be az alábbi változókat [Set the main variables]

% Szerző [Author]
\newcommand{\gpkszerzoVezeteknev}{Mogi}
\newcommand{\gpkszerzoKeresztnev}{Róbert}
\newcommand{\neptun}{ABC123}

% Konzulens 1 [Consulent 1]
\newcommand{\gpkkonzulensAMegszolitas}{dr.~}
\newcommand{\gpkkonzulensAVezeteknev}{Konzulens}
\newcommand{\gpkkonzulensAKeresztnev}{Egy}
\newcommand{\gpkkonzulensABeosztas}{mestertanár}

% Konzulens 2 [Consulent 2], ha nincs hagyd üresen
\newcommand{\gpkkonzulensBMegszolitas}{}
\newcommand{\gpkkonzulensBVezeteknev}{Konzulens}
\newcommand{\gpkkonzulensBKeresztnev}{Kettő}
\newcommand{\gpkkonzulensBBeosztas}{okl. gépészmérnök}

% Konzulens 3 [Consulent 3], ha nincs hagyd üresen 
\newcommand{\gpkkonzulensCMegszolitas}{}
\newcommand{\gpkkonzulensCVezeteknev}{}
\newcommand{\gpkkonzulensCKeresztnev}{}
\newcommand{\gpkkonzulensCBeosztas}{}

% Témavezető
\newcommand{\gpktemavezetoMegszolitas}{dr.~}
\newcommand{\gpktemavezetoVezeteknev}{Témavezető}
\newcommand{\gpktemavezetoKeresztnev}{Egy}
\newcommand{\gpktemavezetoBeosztas}{adjunktus}

%TODO Cím, tanszék, dolgozat típusa [Title, department, document type]

\newcommand{\gpkcim}{Kognitív etorobotika} % Cím [Title]
\newcommand{\gpktanszek}{\bmemogi} % Tanszék [Department]
% \bmeatt \bmeara \bmeenergia \bmeepget \bmegt3 \bmemanuf \bmehds \bmemogi \bmemm \bmept
\newcommand{\gpkdoktipus}{\bsc} % Dokumentum típusa [Document type] (\bsc vagy \msc)
\newcommand{\gpkdoktipusrag}{\bscrag} % Ragozás (\bscrag vagy \mscrag)


\newcommand{\gpkmunkatipus}{\gpkdoktipus}       % szakdolgozat/diplomaterv
\newcommand{\gpkmunkatipust}{\gpkmunkatipus\gpkdoktipusrag t}   % tárgyraggal
\newcommand{\gpkmunkatipusok}{\gpkmunkatipus\gpkdoktipusrag k}  % többesszámban

\input{include/tdk-variables}
\newcommand{\szerzoMeta}{\gpkszerzoVezeteknev{} \gpkszerzoKeresztnev} % egy szerző esetén
%\newcommand{\szerzoMeta}{\gpkszerzoVezeteknev{} \gpkszerzoKeresztnev, \tdkszerzoB} % két szerző esetén

%TODO Nyelv kiválasztása [Language configuration -- choose one]
% Beállítások magyar nyelvű dolgozathoz
%--------------------------------------------------------------------------------------
% Elnevezések
%--------------------------------------------------------------------------------------
\newcommand{\bme}{Budapesti Műszaki és Gazdaságtudományi Egyetem}
\newcommand{\gpk}{Gépészmérnöki Kar}

\newcommand{\bmemogi}{Mechatronika, Optika és Gépészeti Informatika Tanszék}

\newcommand{\keszitette}{Készítette}
\newcommand{\konzulens}{Konzulens}
\newcommand{\temavezeto}{Témavezető}

\newcommand{\bsc}{Szakdolgozat}
\newcommand{\bscrag}{o}                 % szakdolgozat[o]t, szakdolgozat[o]k 
\newcommand{\msc}{Diplomaterv}
\newcommand{\mscrag}{e}                 % diplomaterv[e]t, diplomaterv[e]k
\newcommand{\tdk}{TDK dolgozat}
\newcommand{\bsconlab}{BSc Önálló laboratórium}
\newcommand{\msconlabi}{MSc Önálló laboratórium 1.}
\newcommand{\msconlabii}{MSc Önálló laboratórium 2.}

\newcommand{\pelda}{Példa}
\newcommand{\definicio}{Definíció}
\newcommand{\tetel}{Tétel}

\newcommand{\bevezetes}{Bevezetés}
\newcommand{\koszonetnyilvanitas}{Köszönetnyilvánítás}
\newcommand{\fuggelek}{Függelék}

% Opcionálisan átnevezhető címek
\addto\captionsmagyar{%
\renewcommand{\listfigurename}{Illusztrációk jegyzéke}
%\renewcommand{\listtablename}{Saját táblázatjegyzék cím}
%\renewcommand{\bibname}{Saját irodalomjegyzék név}
}

\newcommand{\szerzo}{\gpkszerzoVezeteknev{} \gpkszerzoKeresztnev}
\newcommand{\gpkkonzulensA}{\gpkkonzulensAMegszolitas\gpkkonzulensAVezeteknev{} \gpkkonzulensAKeresztnev}
\newcommand{\gpkkonzulensB}{\gpkkonzulensBMegszolitas\gpkkonzulensBVezeteknev{} \gpkkonzulensBKeresztnev}
\newcommand{\gpkkonzulensC}{\gpkkonzulensCMegszolitas\gpkkonzulensCVezeteknev{} \gpkkonzulensCKeresztnev}
\newcommand{\gpktemavezeto}{\gpktemavezetoMegszolitas\gpktemavezetoVezeteknev{}
\gpktemavezetoKeresztnev}

\newcommand{\selectthesislanguage}{\selecthungarian}

\bibliographystyle{huplain}

\def\lstlistingname{lista}

\newcommand{\appendixnumber}{6}  % a fofejezet-szamlalo az angol ABC 6. betuje (F) lesz

% Settings for English documents
%%--------------------------------------------------------------------------------------
% Elnevezések
%--------------------------------------------------------------------------------------
\newcommand{\bme}{Budapest University of Technology and Economics}
\newcommand{\gpk}{Faculty of Mechanical Engineering}

\newcommand{\bmeatt}{Department of Materials Science and Engineering}
\newcommand{\bmeara}{Department of Fluid Mechanics}
\newcommand{\bmeenergia}{Department of Energy Engineering}
\newcommand{\bmeepget}{Department of Building Service Engineering and Process Engineering}
\newcommand{\bmegtharom}{Department of Machine and Product Design}
\newcommand{\bmemanuf}{Department of Manufacturing Science and Engineering}
\newcommand{\bmehds}{Department of Hydrodynamic Systems}
\newcommand{\bmemogi}{Department of Mechatornics, Optics and Mechanical Engineering Informatics}
\newcommand{\bmemm}{Department of Applied Mechanics}
\newcommand{\bmept}{Department of Polymer Engineering}

\newcommand{\keszitette}{Author}
\newcommand{\konzulens}{Advisor}
\newcommand{\temavezeto}{Témavezető (in english)}

\newcommand{\bsc}{Bachelor's Thesis}
\newcommand{\bsck}{Bachelor's Theses}
\newcommand{\bsct}{Bachelor's Thesis}
\newcommand{\bscHU}{Szakdolgozat}
\newcommand{\bsckHU}{Szakdolgozatok}
\newcommand{\bsctHU}{Szakdolgozatot}
\newcommand{\msc}{Master's Thesis}
\newcommand{\msct}{Master's Thesis}
\newcommand{\msck}{Master's Theses}
\newcommand{\mscHU}{Diplomaterv}
\newcommand{\msckHU}{Diplomatervek}
\newcommand{\msctHU}{Diplomatervet}
\newcommand{\tdk}{Scientific Students' Association Report}
\newcommand{\bsconlab}{BSc Project Laboratory}
\newcommand{\msconlabi}{MSc Project Laboratory 1}
\newcommand{\msconlabii}{MSc Project Laboratory 2}

\newcommand{\pelda}{Example}
\newcommand{\definicio}{Definition}
\newcommand{\tetel}{Theorem}

\newcommand{\eloszo}{Abstract}
\newcommand{\bevezetes}{Introduction}
\newcommand{\koszonetnyilvanitas}{Acknowledgements}
\newcommand{\fuggelek}{Appendix}
\newcommand{\melleklet}{Annex}

% Optional custom titles
%\addto\captionsenglish{%
%\renewcommand*{\listfigurename}{Your list of figures title}
%\renewcommand*{\listtablename}{Your list of tables title}
%\renewcommand*{\bibname}{Your bibliography title}
%}

\newcommand{\szerzo}{\gpkszerzoKeresztnev{} \gpkszerzoVezeteknev}
\newcommand{\gpkkonzulensA}{\gpkkonzulensAMegszolitas\gpkkonzulensAKeresztnev{} \gpkkonzulensAVezeteknev}
\newcommand{\gpkkonzulensB}{\gpkkonzulensBMegszolitas\gpkkonzulensBKeresztnev{} \gpkkonzulensBVezeteknev}
\newcommand{\gpkkonzulensC}{\gpkkonzulensCMegszolitas\gpkkonzulensCKeresztnev{} \gpkkonzulensCVezeteknev}
\newcommand{\gpktemavezeto}{\gpktemavezetoMegszolitas\gpktemavezetoKeresztnev{} \gpktemavezetoVezeteknev}

\newcommand{\selectthesislanguage}{\selectenglish}

\bibliographystyle{plainnat}

\newcommand{\ie}{i.e.\@\xspace}
\newcommand{\Ie}{I.e.\@\xspace}
\newcommand{\eg}{e.g.\@\xspace}
\newcommand{\Eg}{E.g.\@\xspace}
\newcommand{\etal}{et al.\@\xspace}
\newcommand{\etc}{etc.\@\xspace}
\newcommand{\vs}{vs.\@\xspace}
\newcommand{\viz}{viz.\@\xspace} % videlicet
\newcommand{\cf}{cf.\@\xspace} % confer
\newcommand{\Cf}{Cf.\@\xspace}
\newcommand{\wrt}{w.r.t.\@\xspace} % with respect to
\newcommand{\approximately}{approx.\@\xspace}

\newcommand{\appendixletter}{1}  % a fofejezet-szamlalo az angol ABC 1. betuje (A) lesz
\newcommand{\annexletter}{2} % B betű


%--------------------------------------------------------------------------------------
% Page layout setup
%--------------------------------------------------------------------------------------
% we need to redefine the pagestyle plain
% another possibility is to use the body of this command without \fancypagestyle
% and use \pagestyle{fancy} but in that case the special pages
% (like the ToC, the References, and the Chapter pages)remain in plane style

\pagestyle{plain}
\geometry{inner=30mm, outer=20mm, top=20mm, bottom=30mm}


%--------------------------------------------------------------------------------------
% Text and paragraph styling
%--------------------------------------------------------------------------------------

\sectionfont{\Large\upshape\bfseries}  % Section title font
\subsectionfont{\Large\itshape\mdseries}
\subsubsectionfont{\large\itshape\mdseries}
\setcounter{secnumdepth}{3}             % Section numbering depth

\sloppy                                 % Prevent text from spilling over the margin
\widowpenalty=10000 \clubpenalty=10000  % Prevent widow and oprhan rows
\def\hyph{-\penalty0\hskip0pt\relax}    % Enable hyphenation

\onehalfspacing                         % 1.5x Line spacing

% Text setup for Hungarian text
\newcommand{\selecthungarian}{
	\selectlanguage{magyar}
	\setlength{\parindent}{2em}			% Paragraph indentation
	\setlength{\parskip}{5pt}			% Paragraph spacing
	\frenchspacing
}

% Text setup for English text
\newcommand{\selectenglish}{
	\selectlanguage{english}
	\setlength{\parindent}{0em}
	\setlength{\parskip}{8pt}
	\nonfrenchspacing
}

%--------------------------------------------------------------------------------------
% Setup hyperref package
%--------------------------------------------------------------------------------------
\hypersetup{
    % bookmarks=true,            % show bookmarks bar?
    unicode=true,                % non-Latin characters in Acrobat's bookmarks
    pdfnewwindow=true,           % links in new window
    colorlinks=true,             % false: boxed links; true: colored links
    linkcolor=black,             % color of internal links
    citecolor=black,             % color of links to bibliography
    filecolor=black,             % color of file links
    urlcolor=black               % color of external links
}

%--------------------------------------------------------------------------------------
% Apply variables
%--------------------------------------------------------------------------------------
% This command is called in the main tex file and uses variables set there.
\newcommand{\applyvariables}{
	\author{\authorName}
	\title{\thesisTitle}

	\hypersetup{
		pdftitle={\thesisTitle},     % title
		pdfauthor={\authorName},     % author
		pdfsubject={\gpkmunkatipus}, % subject of the document
		pdfkeywords={\keywords},     % list of keywords (separate then by comma)
		pdfproducer={\authorName},   % producer of the document (organization)
		pdfcreator={LaTeX}           % creator of the document (application)
	}
}

%--------------------------------------------------------------------------------------
% Set up listings
%--------------------------------------------------------------------------------------
\definecolor{lightgray}{rgb}{0.95,0.95,0.95}
\lstset{
	basicstyle=\scriptsize\ttfamily, % print whole listing small
	keywordstyle=\color{black}\bfseries, % bold black keywords
	identifierstyle=, % nothing happens
	% default behavior: comments in italic, to change use
	% commentstyle=\color{green}, % for e.g. green comments
	stringstyle=\scriptsize,
	showstringspaces=false, % no special string spaces
	aboveskip=3pt,
	belowskip=3pt,
	backgroundcolor=\color{lightgray},
	columns=flexible,
	keepspaces=true,
	escapeinside={(*@}{@*)},
	captionpos=b,
	breaklines=true,
	frame=single,
	float=!ht,
	tabsize=2,
	literate=*
		{á}{{\'a}}1	{é}{{\'e}}1	{í}{{\'i}}1	{ó}{{\'o}}1	{ö}{{\"o}}1	{ő}{{\H{o}}}1	{ú}{{\'u}}1	{ü}{{\"u}}1	{ű}{{\H{u}}}1
		{Á}{{\'A}}1	{É}{{\'E}}1	{Í}{{\'I}}1	{Ó}{{\'O}}1	{Ö}{{\"O}}1	{Ő}{{\H{O}}}1	{Ú}{{\'U}}1	{Ü}{{\"U}}1	{Ű}{{\H{U}}}1
}


%--------------------------------------------------------------------------------------
% Set up theorem-like environments
%--------------------------------------------------------------------------------------
% Using ntheorem package -- see http://www.math.washington.edu/tex-archive/macros/latex/contrib/ntheorem/ntheorem.pdf

\theoremstyle{plain}
\theoremseparator{.}
\newtheorem{example}{\pelda}

\theoremseparator{.}
%\theoremprework{\bigskip\hrule\medskip}
%\theorempostwork{\hrule\bigskip}
\theorembodyfont{\upshape}
\theoremsymbol{{\large \ensuremath{\centerdot}}}
\newtheorem{definition}{\definicio}

\theoremseparator{.}
%\theoremprework{\bigskip\hrule\medskip}
%\theorempostwork{\hrule\bigskip}
\newtheorem{theorem}{\tetel}


%--------------------------------------------------------------------------------------
% Some new commands and declarations
%--------------------------------------------------------------------------------------
\newcommand{\code}[1]{{\upshape\ttfamily\scriptsize\indent #1}}
\newcommand{\doi}[1]{DOI: \href{http://dx.doi.org/\detokenize{#1}}{\raggedright{\texttt{\detokenize{#1}}}}} % A hivatkozások közt így könnyebb DOI-t megadni.

\DeclareMathOperator*{\argmax}{arg\,max}
%\DeclareMathOperator*[1]{\floor}{arg\,max}
\DeclareMathOperator{\sign}{sgn}
\DeclareMathOperator{\rot}{rot}


%--------------------------------------------------------------------------------------
% Setup captions
%--------------------------------------------------------------------------------------
\captionsetup[figure]{
	width=.75\textwidth,
	aboveskip=10pt}

\renewcommand{\captionlabelfont}{\it}
\renewcommand{\captionfont}{\footnotesize\it}

%--------------------------------------------------------------------------------------
% Hyphenation exceptions
%--------------------------------------------------------------------------------------
\hyphenation{Shakes-peare Mar-seilles ár-víz-tű-rő tü-kör-fú-ró-gép}


%--------------------------------------------------------------------------------------
% Command to exclude tables and images in the annex from the List of Figures/Tables
%--------------------------------------------------------------------------------------
\newcommand{\excludeFromLocAndLot}{
	% Redefine \addcontentsline to be silent when printing loc or toc entries
	\let\svaddcontentsline\addcontentsline
	\renewcommand\addcontentsline[3]{
		\ifthenelse{\equal{##1}{lof}}{}{
			\ifthenelse{\equal{##1}{lot}}{}{
				\svaddcontentsline{##1}{##2}{##3}
			}
		}
	}
}


%--------------------------------------------------------------------------------------
% Dokumentum törzse [Document body]
%--------------------------------------------------------------------------------------
\begin{document}

\pagenumbering{gobble}
\selectthesislanguage


%TODO Címoldal -- válassz egyet [Titlepage -- choose one from below]
\hypersetup{pageanchor=false}

%--------------------------------------------------------------------------------------
% Szennycímoldal [Cover title page]
%--------------------------------------------------------------------------------------

\clearpage
\begin{center}
\MakeUppercase{\szerzo}\\[0.1cm]
\MakeUppercase{\gpkdoktipus}\\[0.1cm]
\end{center}
\thispagestyle{empty}

%--------------------------------------------------------------------------------------
% Sorozatcímoldal [Series title page]
%--------------------------------------------------------------------------------------
\clearpage
\begin{center}

\includegraphics[width=60mm,keepaspectratio]{figures/bme_logo.pdf}\\
\vspace{0.3cm}
\MakeUppercase{\textbf{\bme}}\\[0.1cm]
\MakeUppercase{\textmd{\gpk}}\\[0.1cm]
\MakeUppercase{\textmd{\gpktanszek}}\\[0.5cm]

\includegraphics[height=40mm,keepaspectratio]{figures/tanszek_logo.pdf}\\[0.5cm]
\MakeUppercase{\gpkmunkatipusok}

\end{center}
\thispagestyle{empty}

%--------------------------------------------------------------------------------------
%	Címoldal [Title page]
%--------------------------------------------------------------------------------------
\begin{titlepage}
\begin{center}
\includegraphics[width=60mm,keepaspectratio]{figures/bme_logo.pdf}\\
\vspace{0.3cm}
\MakeUppercase{\textbf{\bme}}\\[0.1cm]
\MakeUppercase{\textmd{\gpk}}\\[0.1cm]
\MakeUppercase{\textmd{\gpktanszek}}\\[5cm]

\vspace{0.4cm}
{\huge \textsc{\szerzo}}\\[0.8cm]
{\huge \MakeUppercase{\gpkdoktipus}}\\[0.8cm]
{\LARGE \gpkcim}\\[4cm]

{
	\renewcommand{\arraystretch}{0.85}
	\begin{tabular}{ll}
	 \makebox[7cm][l]{\konzulens:} & \makebox[7cm][l]{\temavezeto:} \\
	 \noalign{\smallskip}
	 \makebox[7cm][l]{\hspace{1cm}\emph{\gpkkonzulensA}} & \makebox[7cm][l]{\hspace{1cm}\emph{\gpktemavezeto}} \\
	 \makebox[7cm][l]{\hspace{1cm}\gpkkonzulensABeosztas} & \makebox[7cm][l]{\hspace{1cm}\gpktemavezetoBeosztas} \\
	 \\
	 \makebox[7cm][l]{\hspace{1cm}\emph{\gpkkonzulensB}} & \\
	 \makebox[7cm][l]{\hspace{1cm}\gpkkonzulensBBeosztas} & \\
	 
	\end{tabular}
}

\vfill
{\LARGE Budapest, \the\year.}
\end{center}
\end{titlepage}
\hypersetup{pageanchor=false}
\thispagestyle{empty}
		      % Szakdolgozat/Diplomaterv címlap [Thesis]
%%% TDK címlap
\begin{titlepage}
  \begin{center}  
  \includegraphics[width=7cm]{./figures/bme_logo.pdf}
  \vspace{0.3cm}
  
  \bme \\
  \gpk \\
  \gpktanszek \\
  \vspace{5cm}
  
  \huge {\gpkcim}
  \vspace{1.5cm}
  
  \large {\textbf{\tdk}}
  \vfill
    
  {\Large 
  	\keszitette: \\ \vspace{0.3cm}
  	\szerzo \\
	\tdkszerzoB \\
  	\vspace{1.5cm}
  	\konzulens: \\ \vspace{0.3cm}
  	\gpkkonzulensA \\
  	\gpkkonzulensB \\
  }
  
  \vspace{2cm}
  \large {\tdkev}
 \end{center}
\end{titlepage}
%% Címlap vége
	  % TDK címlap [TDK]
%\include{include/titlepage-otdk}   % OTDK címlap [OTDK]


% Copyright page
\selectlanguage{magyar}
\pagenumbering{gobble}
\selecthungarian
%--------------------------------------------------------------------------------------
% Copyrightoldal
%--------------------------------------------------------------------------------------
\begin{flushleft}
Szerzői jog {\textcopyright} \szerzo, \the\year.
\end{flushleft}

\vfill
\clearpage
\thispagestyle{empty} % an empty page

\selectthesislanguage
     % Copytightoldal [Copyright page]


% Feladatkiírás [Project page]
%TODO A nyomtatott verzóban ne szerepeljen! [Remove before printing]
%--------------------------------------------------------------------------------------
% Feladatkiiras (a tanszeken atveheto, kinyomtatott valtozat)
%--------------------------------------------------------------------------------------
\begin{center}
\large
\textbf{FELADATKIÍRÁS}\\
\end{center}
\thispagestyle{empty}

A feladatkiírást a tanszéki adminisztrációban lehet átvenni, és a leadott munkába eredeti, tanszéki pecséttel ellátott és a tanszékvezető által aláírt lapot kell belefűzni (ezen oldal \emph{helyett}, ez az oldal csak útmutatás). Az elektronikusan feltöltött dolgozatban már nem kell beleszerkeszteni ezt a feladatkiírást.



% Nyilatkozatok [Declarations]
\selectlanguage{magyar}
\selecthungarian
\pagenumbering{roman}
\setcounter{page}{6}
\cleardoublepage % duplexnél páratlan oldalon legyen
%--------------------------------------------------------------------------------------
% Nyilatkozatok
%--------------------------------------------------------------------------------------
\begin{center}
\section*{NYILATKOZATOK}
\end{center}

\vspace{0.5cm}

\begin{center}
\emph{Beadhatósági nyilatkozat}
\end{center}
A jelen \MakeLowercase{\gpkmunkatipusHU} az üzem/intézmény által elvárt szakmai színvonalnak mind tartalmilag, mind formailag megfelel, beadható.

\begin{flushleft}
Kelt,
\end{flushleft}

\begin{flushright}
 \makebox[7cm][l]{Az üzem részéről:}\\
 \vspace{0.5cm}
 \makebox[7cm]{\rule{6cm}{.4pt}}\\
 \makebox[7cm]{\emph{üzemi konzulens}}
\end{flushright}
\vspace{0.6cm}

%--------------------------------------------------------------------------------------

\begin{center}
\emph{Elfogadási nyilatkozat}
\end{center}
Ezen \MakeLowercase{\gpkmunkatipusHU} a Budapesti Műszaki és Gazdaságtudományi Egyetem Gépészmérnöki Kara által a Diplomatervezési és Szakdolgozat feladatokra előírt valamennyi tartalmi és formai követelménynek, továbbá a feladatkiírásban előírtaknak maradéktalanul eleget tesz. E \MakeLowercase{\gpkmunkatipustHU} a nyilvános bírálatra és nyilvános előadásra alkalmasnak tartom.

\begin{flushleft}
A beadás időpontja:
\end{flushleft}

\begin{flushright}
 \makebox[7cm]{\rule{6cm}{.4pt}}\\
 \makebox[7cm]{\emph{témavezető}}
\end{flushright}
\vspace{0.6cm}

%--------------------------------------------------------------------------------------

\begin{center}
\emph{Nyilatkozat az önálló munkáról}
\end{center}
Alulírott,  \emph{\authorFamilyName{} \authorGivenName} (\neptun), a Budapesti Műszaki és Gazdaságtudományi Egyetem hallgatója, büntetőjogi és fegyelmi felelősségem tudatában kijelentem és sajátkezű aláírásommal igazolom, hogy ezt a \MakeLowercase{\gpkmunkatipustHU} meg nem engedett segítség nélkül, saját magam készítettem, és dolgozatomban csak a megadott forrásokat használtam fel. Minden olyan részt, melyet szó szerint vagy azonos értelemben, de átfogalmazva más forrásból átvettem, egyértelműen, a hatályos előírásoknak megfelelően, a forrás megadásával megjelöltem.

\begin{flushleft}
Budapest, \today
\end{flushleft}

\begin{flushright}
 \makebox[7cm]{\rule{6cm}{.4pt}}\\
 \makebox[7cm]{\emph{hallgató}}
\end{flushright}


\vfill
\clearpage

\selectthesislanguage

\newcounter{romanPage}
\setcounter{romanPage}{\value{page}}
\stepcounter{romanPage}



% Tartalomjegyzék [Table of Contents]
\tableofcontents\vfill


% Előszó [Preface]


\selecthungarian

%----------------------------------------------------------------------------
\chapter*{\eloszo}\addcontentsline{toc}{chapter}{\eloszo}
%----------------------------------------------------------------------------

Az előszó legtöbbször személyes hangú, eligazító jellegű írás, amely a mű megírásának okairól, születésének körülményeiről szól. Az előszó nem szerves része a főszövegnek, hanem annak kiegészítése.
Ugyancsak az előszóban fejtheti ki a szerző a mű megértéséhez szükséges szempontokat, a követett módszereket, utalhat a fontosabbelőzményekre és szakirodalomra.
Az előszó ne legyen terjedelmes.

\begin{flushleft}
Jelen dokumentum egy diplomaterv-sablon, amely formai keretet ad a BME Gépészmérnöki Karán végző hallgatók által elkészítendő szakdolgozatnak és diplomatervnek. A sablon használata opcionális. Ez a sablon \LaTeX~alapú, a \emph{TeXLive} \TeX-implementációval és a PDF-\LaTeX~fordítóval működőképes.

A sablon forrása a \verb+https://github.com/leventerevesz/bme-gpk-thesis-latex+ repository-ban elérhető.  Amennyiben hibát találtál, vagy észrevételed, javaslatod lenne, kérlek ott jelezd.
\end{flushleft}


\begin{center}
    $\sim \: \sim \: \sim$
\end{center}

\begin{flushleft}
Köszönetnyilvánítás \emph{A köszönetnyilvánítás ide írható.} Jelen sablon a Villamosmérnöki és Informatikai Kar Méréstechnika és Információs Rendszerek Tanszék szakdolgozat és diplomaterv sablonja alapján készült. Köszönöm készítőinek az áldozatos munkájukat.
\end{flushleft}

\vspace{0.5cm}

\begin{flushleft}
{Budapest, \today}
\end{flushleft}

\begin{flushright}
\emph{\szerzo}
\end{flushright}

\vfill
           % GPK sablon szerint


% Jelölések jegyzéke [Table of Symbols]
\newcommand{\tss}{\textsuperscript}     % tss = felső index
%----------------------------------------------------------------------------
\chapter*{Jelölések jegyzéke}\addcontentsline{toc}{chapter}{Jelölések jegyzéke}
%----------------------------------------------------------------------------

A táblázatban a többször előforduló jelölések magyar és angol nyelvű elnevezése, 
valamint a fizikai mennyiségek esetén annak mértékegysége található. Az egyes 
mennyiségek jelölése – ahol lehetséges – megegyezik hazai és a nemzetközi 
szakirodalomban elfogadott jelölésekkel. A ritkán alkalmazott jelölések 
magyarázata első előfordulási helyüknél található.

%%~~~~~~~~~~~~~~~~~~~~~~~~~~~~~~~~~~~~~~~~~~~~~~~~~~~~~~~~~~~~~~~~~~~~~~~~~~~~~~~~~~~~~
%% A táblázatokat ABC rendben kell feltölteni, először mindig a kisbetűvel kezdve. Ha
%% egyazon betűjelnek több értelmezése is van, akkor mindegyiket külön sorban kell fel-
%% tüntetni. Konstansok esetén az értéket is a táblázatba kell írni. Dimenzió nélküli
%% mennyiségek mértékegysége 1 és nem: – ! A jelölésjegyzékben csak SI vagy SI-n kí-
%% vüli engedélyezett mértékegységeket szabad feltüntetni. Egy dokumentumon belül az
%% SI és pl. az angolszász mértékrendszer nem keverhető!
%%~~~~~~~~~~~~~~~~~~~~~~~~~~~~~~~~~~~~~~~~~~~~~~~~~~~~~~~~~~~~~~~~~~~~~~~~~~~~~~~~~~~~~

%%~~~~~~~~~~~~~~~~~~~~~~~~~~~~~~~~~~~~~~~~~~~~~~~~~~~~~~~~~~~~~~~~~~~~~~~~~~~~~~~~~~~~~
%% A Jelölés oszlop alapvetően kurzív betűváltozattal szedendő, a Mértékegység oszlo-
%% pot álló betűkkel kell szedni. Felső indexhez használható a \tss{} parancs.
%%~~~~~~~~~~~~~~~~~~~~~~~~~~~~~~~~~~~~~~~~~~~~~~~~~~~~~~~~~~~~~~~~~~~~~~~~~~~~~~~~~~~~~

\def\arraystretch{1.5}%  vertical cell padding

\subsection*{Latin betűk}
\begin{center}
    \begin{tabular}{lp{10cm}l}
        \hline
        \large{Jelölés} & \large{Megnevezés, megjegyzés, érték} & \large{Mértékegység} \\ 
        \hline
        $g$     & gravitációs gyorsulás (9.81)  & m/s\tss{2}     \\
        $p$     & nyomás                        & bar           \\
        $s$     & fajlagos entrópia             & J/(kg$\cdot$K)\\
        \hline
    \end{tabular}    
\end{center}



\subsection*{Görög betűk}
\begin{center}
    \begin{tabular}{lp{10cm}l}
        \hline
        \large{Jelölés} & \large{Megnevezés, megjegyzés, érték} & \large{Mértékegység} \\ 
        \hline
        $\eta$  & hatásfok                      & 1             \\      
        $\rho$  & sűrűség                       & kg/m\tss{3}    \\
        \hline
    \end{tabular}
\end{center}



\subsection*{Indexek, kitevők}
\begin{center}
    \begin{tabular}{lp{12.8cm}}
        \hline
        \large{Jelölés} & \large{Megnevezés, értelmezés}\\ 
        \hline
        $i$     & általános futóindex (egész szám)  \\
        nom     & névleges (nominális) érték        \\
        opt     & legkedvezőbb (optimális) érték    \\
        \hline
    \end{tabular}    
\end{center}


\def\arraystretch{1}%  vertical cell padding


% The main part of the thesis
\pagenumbering{arabic}
%TODO Importáld a saját fejezeteidet [Import your own content]
\include{content/introduction}      % Bevezetés
\include{content/latex-tools}       % 1. fejezet
\include{content/thesis-format}     % 2. fejezet
\include{content/template-usage}    % 3. fejezet
%----------------------------------------------------------------------------
\chapter{\osszefoglalas} % (Eredmények értékelése)
%----------------------------------------------------------------------------


\section{Eredmények}
Az összefoglaló értékelés a három oldalt lehetőleg ne haladja meg! 
Az elvégzett munka és eredményeinek bemutatása egyes szám első személyben fogalmazva.


\section{Javaslatok/Következtetések/Tanulságok} % Válassz egyet
A feladat elkészítése során levont tanulságok összefoglalása. Javaslattétel, 
továbbfejlesztési lehetősége bemutatása, előretekintés a jövőbe stb.

% Keltezés, aláírás
\vspace{0.5cm}

\begin{flushleft}
{Budapest, \today}
\end{flushleft}

\begin{flushright}
\emph{\authorName}
\end{flushright}

\vfill
           % Összefoglalás


% Bibliográfia [Bibliography]
\addcontentsline{toc}{chapter}{\bibname}
\bibliography{bib/mybib}


% Idegen nyelvű (angol) összefoglaló [Foreign language summary]
%----------------------------------------------------------------------------
\chapter*{\summary}\addcontentsline{toc}{chapter}{\summary}
%----------------------------------------------------------------------------

Az elvégzett munka rövid, másfél oldalt meg nem haladó, de legalább 2/3 
oldalnyi terjedelmű angol nyelvű összefoglalása.

Angol nyelven készített dolgozat esetén magyar nyelvű összefoglaló kell, ha a
készítő magyar anyanyelvű. Nem angol vagy nem magyar nyelven készített
dolgozat esetén kötelező az angol nyelvű összefoglaló, és ha a készítő magyar
anyanyelvű, akkor a magyar nyelvű is.

\vspace{0.5cm}
Keywords: \emph{keyword1, keyword2, keyword3}
\vspace{0.5cm}



% Függelék és mellékletek [Appendices]
%----------------------------------------------------------------------------
\appendix
%----------------------------------------------------------------------------
\chapter*{\fuggelek}\addcontentsline{toc}{chapter}{\fuggelek}
%----------------------------------------------------------------------------
\setcounter{chapter}{\appendixletter} % F betű
%\setcounter{equation}{0}
\numberwithin{equation}{section}
\numberwithin{figure}{section}
\numberwithin{lstlisting}{section}
%\numberwithin{tabular}{section}


% Ismertető - töröld ki
%----------------------------------------------------------------------------
A függelék a főszöveget kiegészíti további részletezéssel. Ide kerül minden kiegészítő információ, ami nem tartozik szorosan a feladat témájához. A függelék rendszerint nem önálló dokumentum. A főszöveg általában nem hivatkozik rá. Általában saját munka.

A dolgozat opcionális eleme, csak igény esetén kell használni.
%----------------------------------------------------------------------------

%----------------------------------------------------------------------------
\section{A TeXstudio felülete}
%----------------------------------------------------------------------------
\begin{figure}[!ht]
\centering
\includegraphics[width=150mm, keepaspectratio]{figures/TeXstudio.png}
\caption{A TeXstudio \LaTeX-szerkesztő.} 
\end{figure}

%----------------------------------------------------------------------------
\clearpage\section{Válasz az ,,Élet, a világmindenség, meg minden'' kérdésére}
%----------------------------------------------------------------------------
A Pitagorasz-tételből levezetve
\begin{align}
c^2=a^2+b^2=42.
\end{align}
A Faraday-indukciós törvényből levezetve
\begin{align}
\rot E=-\frac{dB}{dt}\hspace{1cm}\longrightarrow \hspace{1cm}
U_i=\oint\limits_\mathbf{L}{\mathbf{E}\mathbf{dl}}=-\frac{d}{dt}\int\limits_A{\mathbf{B}\mathbf{da}}=42.
\end{align}
        % Függelék - opcionális
%----------------------------------------------------------------------------
\chapter*{\melleklet}\addcontentsline{toc}{chapter}{\melleklet}
%----------------------------------------------------------------------------

%----------------------------------------------------------------------------
\section*{1. Melléklet}\addcontentsline{toc}{section}{1. Melléklet}
%----------------------------------------------------------------------------

%----------------------------------------------------------------------------
\clearpage
\section*{2. Melléklet}\addcontentsline{toc}{section}{2. Melléklet}
%----------------------------------------------------------------------------
		    % Mellékletek


% Ábrák és táblázatok jegyzéke [List of Figures, Tables]
\listoffigures\addcontentsline{toc}{chapter}{\listfigurename}   % Ábrák jegyzéke - opcionális
\listoftables\addcontentsline{toc}{chapter}{\listtablename}     % Táblázatok jegyzéke - opcionális


%\label{page:last}
\end{document}
